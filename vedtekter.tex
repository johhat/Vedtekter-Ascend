\documentclass[a4paper,11pt,titlepage]{article}
\usepackage{geometry} %Endrer marginer
\usepackage[T1]{fontenc} 
\usepackage[utf8]{inputenc}
\usepackage{multicol}

\setlength{\parindent}{0em}
\setlength{\parskip}{1em}
\setlength{\columnsep}{1cm}

\newcommand{\orgName}{Ascend}

\begin{document}

\section*{Vedtekter for foreningen Ascend}

\subsection*{§1 Foreningens navn}
Foreningens navn er \orgName.

\subsection*{§2 Formål}

\begin{enumerate}
\item {\orgName} er en frivillig interesseorganisasjon av og for studenter ved ved Norges teknisk-naturvitenskapelige universitet (NTNU).
\item {\orgName} skal fremme NTNU-studenters interesse for autonome ubemannede fartøy, med vekt på utvikling av autonome ubemannede fartøy som kan løse oppgavene i konkurranser som International Aerial Robotics Competition (IARC) og lignende.
\item Foreningen skal være miljøskapende, ved å aktivisere medlemmene faglig og sosialt.
\item All aktiviteten i foreningen skal være av frivillig karakter.
\item Foreningens aktivitet er ikke ment å være profittskapende.
\end{enumerate}

\subsection*{§3 Juridisk person}
Foreningen er selveiende og en frittstående juridisk person med upersonlig og begrenset ansvar for gjeld.

\subsection*{§4 Medlemmer}

\begin{enumerate}
\item Kun nåværende og tidligere studenter og stipendiater ved NTNU kan være medlemmer i foreningen.
\item Kun personer som er godkjent av styret kan bli med i foreningen. Styret kan delegere denne myndigheten til enkeltpersoner i foreningen.
\item Medlemskap er delt inn i aktivt medlemskap og alumnimedlemskap. 
%TODO Dette er for kronglete. "Trer ut av alle roller", hva skal det bety?
\item Nye medlemmer har først aktivt medlemskap. Medlemskapet går over til alumnimedlemskap i det medlemmet trer ut av alle roller i foreningen. 
\item Alumnimedlemene representerer et nettverk av tidligere medlemmer. Alumnimedlemmer har ingen beslutningsmyndighet, men kan bidra som konsulenter på vegne av foreningen etter avtale med styret. Alumnimedlemmer har ikke stemmerett ved generalforsamlingen.
%TODO: Er dette en ok definisjon av tillitsverv?
\item Kun aktive medlemmer i foreningen kan ha tillitsverv. Med tillitsverv menes styremedlemskap og/eller verv som innebærer at man opptrer som andre aktive medlemmers overordnede.
\item Medlemmene plikter å gjøre seg kjent med vedtektene og å overholde vedtekter og bestemmelser.
\item Opphør av medlemskap
\begin{enumerate}
\item Medlemmer som ikke har betalt medlemskontingent innen gitte frister betraktes ikke som medlemmer. Styret fastsetter frister for innbetaling av kontingent.
\item Et medlem kan ekskluderes dersom det har begått grove brudd på vedtektene, eller har opptrådt på en måte som er svært skadelig for foreningen.
\item Styret kan med 2/3 flertall i fulltallig styremøte midlertidig eller permanent permittere medlemmer fra stillinger i alle ledd av foreningen. Ved permittering, kan styret velge å delegere denne personens oppgaver og ansvar til andre medlemmer. Ved permittering opprettholder medlemmet sin status som aktivt medlem.
\item Kun styret har ved 2/3 flertall i fulltallig styremøte anledning til å ekskludere medlemmer fra foreningen.
\end{enumerate}

\end{enumerate}
\subsection*{§5 Stemmerett og valgbarhet}
Alle aktive medlemmer har stemmerett og er valgbare til tillitsverv i foreningen.

\subsection*{§6 Kontingent}
Den årlige kontingenten til foreningen blir vedtatt på generalforsamlingen.

\subsection*{§7 Tillitsvalgtes godtgjørelse}
Tillitsvalgte skal ikke motta honorar for sine verv.

\subsection*{§8 Generalforsamling}
\begin{enumerate}
\item Generalforsamlingen er foreningens årsmøte, og skal holdes hvert år i september eller oktober måned. Generalforsamlingen er foreningens høyeste myndighet.

\item Styret innkaller medlemmene til generalforsamling med minst en ukes varsel. 

\begin{enumerate}
\item Aktive medlemmer skal underrettes via epost eller andre samarbeidsplattformer foreningen bruker. 
%TODO: Skrive denne om til å høres mer positiv ut
\item Styret er ikke pliktig til å kontakte alumnimedlemmer direkte om generalforsamlingen. Innkalling til generalforsamlingen skal tilgjengeliggjøres i offentlige og/eller andre informasjonskanaler hvor alumnimedlemmer har anledning til å oppsøke denne informasjonen selv. 
\end{enumerate}

\item Styrets forslag til vedtektsendinger skal være med i innkallelsen til generalforsamlingen.

\item Forslag som skal behandles på generalforsamlingen skal være sendt til styret senest fem dager før generalforsamlingen. Fullstendig saksliste må være tilgjengelig for medlemmene senest tre dager før generalforsamlingen.

\item Alle medlemmer har adgang til generalforsamlingen. Generalforsamlingen kan invitere andre personer og/eller media til å være til stede.

\item Generalforsamlingen er vedtaksført med det antall stemmeberettigede medlemmer som møter. Ingen har mer enn én stemme, og stemmegivning kan ikke skje ved fullmakt.

\item Generalforsamlingen kan bare behandle forslag om vedtektsendring som er oppført på sakslisten senest tre dager før Generalforsamlingen.

\item Generalforsamlingen kan ikke behandle forslag som ikke er oppført på sakslisten, med mindre 3/4 av de fremmøtte krever det. En slik beslutning kan bare tas i forbindelse med vedtak om godkjenning av saksliste.

\item Generalforsamlingen ledes av valgt ordstyrer. Ordstyreren behøver ikke å være medlem av foreningen. Styret innstiller ordstyrer, og dette godkjennes av generalforsamlingen ved møtets begynnelse. Ordstyrer kan velge å utnevne en referent.

\end{enumerate}

\subsection*{§9 Stemmegivning på generalforsamlingen}

\begin{enumerate}

\item Med mindre annet er bestemt, skal et vedtak, for å være gyldig, være truffet med alminnelig flertall av de avgitte stemmene. Blanke stemmer skal anses som ikke avgitt.

\item Valg foregår skriftlig hvis det foreligger mer enn ett forslag. Bare foreslåtte kandidater kan føres opp på stemmeseddelen. Skal flere velges ved samme avstemming, må stemmesedlene inneholde det antall forskjellige kandidater som det skal velges ved vedkommende avstemming. Stemmesedler som er blanke, eller som inneholder ikke foreslåtte kandidater, eller annet antall kandidater enn det som skal velges, teller ikke, og stemmene anses som ikke avgitt.

\item Når et valg foregår enkeltvis, og en kandidat ikke oppnår mer enn halvparten av de oppgitte stemmene, foretas bundet omvalg mellom de to kandidatene som har oppnådd flest stemmer. Er det ved omvalg stemmelikhet, avgjøres valget ved loddtrekning.

\item Når det ved valg skal velges flere ved en avstemming, må alle, for å anses valgt, ha mer enn halvparten av de avgitte stemmene. Dette gjelder ikke ved valg av vararepresentanter. Hvis ikke tilstrekkelig mange kandidater har oppnådd dette i første omgang, anses de valgt som har fått mer enn halvparten av stemmene. Det foretas så bundet omvalg mellom de øvrige kandidatene, og etter denne avstemmingen anses de valgt som har fått flest stemmer. Er det ved omvalg stemmelikhet, avgjøres valget ved loddtrekning.

\end{enumerate}

\subsection*{§10 Generalforsamlingens oppgaver}
Generalforsamlingen skal:

\begin{enumerate}

\item Behandle foreningens årsmelding
\item Behandle foreningens regnskap
\item Behandle foreningens budsjett
\item Behandle innkomne forslag
\item Fastsette foreningens medlemskontigent

\end{enumerate}

\subsection*{§11 Ekstraordinær generalforsamling}

\begin{enumerate}

\item Ekstraordinær generalforsamling holdes når styret bestemmer det, eller minst halvparten av de stemmeberettigede medlemmene krever det.

\item Det innkalles på samme måte som for ordinære generalforsamlinger. Bakgrunnen for den ekstraordinære generalforsamlingen skal fremkomme av innkallingen.

\item Ekstraordinær generalforsamling kan bare behandle og ta avgjørelse i de sakene som er kunngjort i innkallingen.

\end{enumerate}

\subsection*{§12 Styret}

\begin{enumerate}

\item Styret er høyeste myndighet i tidsrommet mellom generalforsamlingene. Styret velges på ekstraordinær generalforsamling som avholdes i februar eller mars og har valgperiode på ett år, fra og med datoen for overgang til nytt stye. 

\item Dato for overgang til nytt styre skal være en dato i inneværende år, og skal normalt være en dato i august. Datoen skal settes etter hva som er mest hensiktsmessig for foreningens drift. Spesielt skal det tas hensyn til plasseringen av konkurranser. Det avtroppende og det påtroppende styret skal i samråd bli enige om en overgangsdato. Dersom det ikke blir enighet mellom det avtroppende og påtroppende styret, kan det avtroppende styret velge en dato i august. Datoer utenom august kan kun velges dersom dette gjøres i samråd mellom det avtroppende og påtroppende styret.

\item Valg til nytt styre skal skje etter innstilling fra avtroppende styre. Avtroppende styre har myndighet til å foreta nødvendige intervjuer i forbindelse med sin innstilling.

\item Stemmegivning på den ekstraordinære generalforsamlingen hvor det velges nytt styre skjer i henhold til § 9.

\item Det avtroppende styret kan velge å komme med en samlet innstilling. Dersom den ekstraordinære generalforsamlingen ikke oppnår flertall for styrets innstilling skal kandidatene stemmes på enkeltvis. Den ekstraordinære generalforsamlingen kan foreslå andre kandidater enn de styret har innstilt.

\item Foreningen ledes av et styre på minimalt tre og maksimalt seks medlemmer og utgjør foreningens ledelse. Styret består av en styreleder og øvrige styremedlemmer. 

Styret skal:
\begin{enumerate}
\item Iverksette generalforsamlingens bestemmelser.
\item Oppnevne etter behov komiteer/utvalg/personer for spesielle oppgaver og utarbeide instruks for disse.
\item Administrere og føre nødvendig kontroll med foreningens økonomi i henhold til de til enhver tid gjeldende instrukser og bestemmelser.
\item  Representere foreningen utad. Styret skal holde møte når lederen forlanger det eller et flertall av styremedlemmene forlanger det.
\end{enumerate}

\item Styret skal normalt avholde ett styremøte i uken. Styret kan avholde ekstraordinære møter når styrelederen forlanger det eller et flertall av styremedlemmene forlanger det. Styreleder er møtets leder. Ved styreleders fravær utnevner de tilstedeværende styremedlemmene en møteleder.

\item Styret er vedtaksdyktig når minst halvparten av styrets medlemmer er til stede. Vedtak fattes ved at minst halvparten av de deltakende styremedlemmene stemmer for vedtaket. Ved stemmelikhet avgjør styreleders stemme. 

\item Styrets møter er lukket, men kan åpnes dersom styrets leder eller et flertall av styret ønsker det.

\item Kun styreleder kan uttale seg offentlig på vegne av foreningen. Medlemmer kan uttale seg offentlig på vegne av foreningen etter oppfordring fra styreleder.

\end{enumerate}


\subsection*{§13 Signeringsrett}
Alle styrets medlemmer har rett til å forhandle frem avtaler på vegne av foreningen. Alle styrets medlemmer har signeringsrett på vegne av foreningen. Ved signering må minst to av styrets medlemmer skrive sin underskrift. 

\subsection*{§14 Gjeld}
Kun styret kan pådra foreningen gjeld.

\subsection*{§15 Vedtektsendring}
Endringer i disse vedtekter kan bare foretas på ordinær eller ekstraordinær generalforsamling etter å ha vært på sakslisten, og det kreves 2/3 flertall av de avgitte stemmene.

\subsection*{§16 Oppløsning, sammenslutning og deling}

\begin{enumerate}
\item Oppløsning av foreningen kan bare behandles på ordinær generalforsamling. Blir oppløsning vedtatt med minst 2/3 flertall, innkalles det til ekstraordinær generalforsamling en måneder senere. For at oppløsning skal skje, må vedtaket her gjentas med 2/3 flertall. Det kan velges et avviklingsstyre som skal forestå avviklingen. Det ordinære styret kan velges til avviklingsstyre, og får stilling som avviklingsstyre om intet valg foretas. 

%TODO: Kan jo også donere pengene til veldedighet eller noe.
\item Foreningens formue skal etter oppløsning og gjeldsavleggelse tilfalle det formål foreningen arbeider for å fremme, ved at midlene tilfaller i like andeler Institutt for teknisk kybernetikk og Institutt for datateknikk og informasjonsvitenskap ved NTNU.

\item Ingen medlemmer har krav på foreningens midler eller andel av disse.

\item Sammenslutning med andre foreninger eller deling av foreningen anses ikke som oppløsning. Vedtak om sammenslutning/deling og nødvendige vedtektsendringer i tilknytning til dette treffes i samsvar med bestemmelsene om vedtektsendring, jf. § 15. Styret skal i denne forbindelse utarbeide en plan for sammenslutningen/delingen som generalforsamlingen skal stemme over.

\end{enumerate}

\end{document}
